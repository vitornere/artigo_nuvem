\section{Ambiente}

No artigo analisado foi utilizado a ferramenta OpenNebula e a arquitetura FairCPU para uma estratégia baseada em balanceadores de carga (elasticidade horizontal). O OpenNebula é um conjunto de ferramentas de IaaS open source, que pode ser utilizado para geração de uma nuvem privada. No que se refere a nuvem privada, sua infraestrutura opera exclusivamente para uma única organização ou usuário, seja ela gerida internamente ou por um terceiro, e hospedada internamente ou externamente.

O FairCPU é uma arquitetura para provisionamento de máquinas virtuais utilizando
características de processamento~\cite{faircpu:12}. FairCPU utiliza unidades de processamento (UPs) para alocar recursos de CPU para as MVs\footnote{Máquinas Virtuais}, de forma a garantir que o desempenho da MV seja homogeneo, independente da MF subjacente. A UP e a abstração utilizada para representar o poder de processamento de MFs e MVs, e deve ter um valor constante e conhecido (ex. GFLOPS, MIPS, ou outra metrica) e substitui o valor bruto da quantidade de CPUs, que é o parâmetro utilizado na maioria dos middleware e atuais provedores de IaaS no momento de alocar as MVs.

Essa ferramenta foi utilizada no artigo \cite{coutinho_et_al:14} proposto por possibilitar a simulação em nuvens computacionais para avaliar em conjunto ao FairCPU o comportamento do ambiente diante de cargas de trabalho, e como se comportam de maneira autonomica para adaptação a variações de demanda (elasticidade) e manutencção do SLA.

Nesse artigo foram utilizadas as ferramentas \textit{CloudStack}, na versão 4.4, para simulação e geração de nuvens computacionais e o \textit{VMWare}\footnote{a.k.a Virtual Machine Ware} para simulação de virtualização. O \textit{CloudStack}
é uma ferramente de software de código aberto para computação em nuvem que implementa o que é comumente referido como infraestrutura como serviço (IaaS), que são sistemas que dão aos usuários a capacidade de executar e controlar instâncias de máquinas virtuais inteiras implantados através de uma variedade de recursos físicos~\cite{nurmi_2009}. 

A \textit{VMWare} foi utilizada para a criação de uma máquina virtual do sistema operacional Ubunto Server 14.04.1 para replicação do experimento ocorrido no \cite{elaine_et_al:14}. O conceito de máquina virtual é definido pela IBM como: uma cópia do hardware físico da máquina totalmente protegido e isolado. Assim, cada usuário de uma máquina virtual possui a ilusão de ter uma máquina física dedicada. Os desenvolvedores de software também podem escrever e testar programas sem o medo de deixar que a máquina física
não funcione e afete outros usuários. \cite{sugerman2001virtualizing}.
