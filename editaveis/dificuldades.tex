\section{Dificuldades}
\label{sec:dificuldades}

Este artigo foi feito baseado na atividade de refazer o experimento no artigo \cite{coutinho_et_al:14} utilizando de outra tecnologia que não fosse usada no artigo alvo. O artigo alvo usa a ferramenta \textit{OpenNebula}. Já o experimento de replicação está usando o \textit{ClouStack}. Este experimento foi feito para a matéria de Computação em Nuvem da Faculdade do Gama.

Baseado neste fato, alguns problemas relacionados a esta adaptação foram encontrados.

\begin{enumerate}

  \item Instalação da ferramenta CloudStack

  \begin{itemize}
    \item Problema: Foi feita a escolha de instalar o CloudStack em uma máquina virtual com o Sistema Operacional Mint 17.1, porém não existe, atualmente, as dependências que necessitam ser instaladas.
    \item Solução: Houve a troca do Sistema Operacional para Ubunto Server 14.04.1.
  \end{itemize}

  \item Simulação de máquinas virtuais
  \begin{itemize}
    \item Problema: O CloudStack não faz simulação, somente a virtualização de ambiente como nuvem.
    \item Solução: Utilização de máquinas virtuais em baixa quantidade.
  \end{itemize}

  \item Simulação da quantidade requerida de máquinas virtuais
  \begin{itemize}
    \item Problema: O computador utilizado como Nó de controle para o CloudStack, não tem poder computacional para virtualização da quantidade mínima de máquinas requerida pelo artigo alvo.
    \item Solução: Utilização de uma quantidade mínima para simulação
  \end{itemize}

  \item Criação de um Host
  \begin{itemize}
    \item Problema: A criação do Host não foi com sucesso.
    \item Solução: Tentar procurar como corrigir o problema e criar um Host. Porém a procura não foi bem sucedida, pois chegou-se que o computador não possui poder computacional suficiente para criar um Host.
  \end{itemize}



\end{enumerate}


\begin{comment}

  \item Instalação da ferramenta CloudStack
  \begin{itemize}
    \item Problema:
    \item Solução:
  \end{itemize}


\end{comment}
