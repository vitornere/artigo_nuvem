\section{Introdução}
\label{sec:introducao}

A Computação em Nuvem propõe a integração de diversos modelos tecnológicos para o
provimento de infraestrutura de hardware, plataformas de desenvolvimento e aplicações
na forma de serviços sob demanda com pagamento baseado em uso \cite{sa:11}. A elasticidade é uma das principais características da Computação em Nuvem. Esta característica consiste na capacidade de adicionar ou remover recursos, sem interrupções e em tempo de execução para lidar com a variação da carga. 

A computação autonômica e inspirada em sistemas biológicos para lidar com desafios de complexidade, dinamismo e heterogeneidade \cite{kephart:03}, características presentes nos ambientes de computação em nuvem e, assim, fornecer uma abordagem promissora neste contexto \cite{sousa:11}. De acordo com \cite{kephart:03}, um sistema autonômico compreende um conjunto de elementos autonômicos. Um elemento deste conjunto é um componente responsável para o gerenciamento de seu próprio comportamento, em conformidade com políticas definidas, e para interação com outros elementos autonômicos que provém ou consomem serviços computacionais.

O monitoramento de recursos computacionais, como CPU e memória, se torna 
essencial tanto para os provedores quanto para clientes. Uma maneira de monitorar
aplicações em nuvem de modo mais efetivo e por meio da utilização de mecanismos de 
Computação Autonômica

Diversas arquiteturas para soluções de provisionamento e manutenção de 
SLA utilizando recursos de Computacao Autonômica foram propostas na literatura 
\cite{rego:12,tordsson:12}. Segundo \cite{coutinho_et_al:14}, devido a grande diversidade de tecnologias e a pouca disponibilidade de informação acerca de instalação e configuração, do ponto de vista experimental em geral não e fácil utilizá-las.

O artigo está estruturado da seguinte forma:
\begin{itemize}
  \item Seção 2 será descrito o ambiente para replicação do artigo \cite{coutinho_et_al:14}.
  \item Seção 3 será descrito um paralelo entre: como o artigo propôs a implementação do mesmo e como o experimento deste artigo foi realizado.
  \item Seção 4 será descrito as dificuldades, problemas e soluções em relação ao desenvolvimento da replicação dos resultados do artigo.
  \item Seção 5 será descrito os resultados.
\end{itemize}
