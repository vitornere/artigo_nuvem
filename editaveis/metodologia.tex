\section{Metodologia}

O objetivo deste artigo, é o da replicação do experimento feito em \cite{coutinho_et_al:14}. Os problemas encontrados no decorrer do experimento foram devidamente citados na seção \ref{sec:dificuldades}.

Para iniciar o experimento de replicação, inicialmente foi feita o levantamento de ambiente, que consistiu basicamente do download do \textit{hipervisor}\footnote{Camada de software localizada entre a camada de hardware e o sistema operacional\cite{dev_midia}} de simulação de virtualização \textit{VMWare}\footnote{a.k.a Virtual Machine Ware} e do download da imagem virtual do sistema operacional \textit{Ubunto Server} onde ficará o \textit{CloudStack}. Esta imagem virtual contém uma imagem do sistema operacional Ubunto 14.04.1 onde o \textit{CloudStack} será posteriormente instalado. Após algumas configuraçoes na máquina virtual, o CloudStack poderá ser iniciado pelo brownser.

Após efetuado login, é necessário configurar o \textit{Zone}, que é equivalente a um único centro de dados, o \textit{Pod}, possui um ou mais \textit{clusters}, o \textit{Cluster}, consiste em um ou mais \textit{hosts} e \textit{storage}, o \textit{Storage}, um recurso de armazenamento normalmente fornecido para um único cluster para o funcionamento real de instância de imagem de disco, e por fim o \textit{Host}, um nó de computação único dentro de um cluster, muitas vezes um \textit{hypervisor}.

Segundo o artigo \cite{coutinho_et_al:14}, para a adaptação dos recursos para a carga de trabalho foi utilizada uma estratégia baseada em balanceadores de carga (elasticidade horizontal), por meio da ferramenta NGINX, que é um servidor leve que poupa o uso exagerado de recursos do servidor, provendo assim uma maior eficiência no uso. No artigo definido foi utilizada carga de trabalho projetada para executar operacões de multiplicação de matrizes atravês de um \textit{microbenchmark} desenvolvido na linguagem de programação Java, e uma aplicação científica, o \textit{BLAST}. Ambas sendo orientadas a CPU e memória e projetadas para serem executadas como pequenas aplicações web, disparadas por meio do \textit{HTTPERF}, conforme uma taxa pré-definida pelo artigo.

Ao final do experimento foi criado um mecanismo de Computação Autonômica baseado na arquitetura definida em [Kephart and Chess 2003]. O objetivo dos experimentos exibidos no artigo é de avaliar o comportamento do ambiente diante de cargas de trabalho, e como se comportam de maneira autonômica para adaptação a variações de demanda,que é a elasticidade, e manutenção da qualidade de serviço.
